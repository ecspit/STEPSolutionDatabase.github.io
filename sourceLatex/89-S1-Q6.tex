\documentclass{article}
\usepackage{graphicx}
\usepackage{amsmath}   
\usepackage{amsthm}
\usepackage{tikz}
\usepackage{pgfplots}
\pgfplotsset{compat=1.18}
\setlength{\parindent}{0pt}

\title{89-S1-Q6 \\ Solution + Discussion}
\author{David Puerta}
\date{}

\begin{document}

\maketitle

\begin{abstract}
    \noindent In this document we will go through the solution to the 89-S1-Q6 question and provide a discussion of the question at the end. There are also hints on the first page to aid you in finding a solution. There is no single method that results in an answer to a STEP question, there are a multitude of different paths that end up at the same solution. However, some methods are more straight forward and you are encouraged to take the path of least resistance.  
\end{abstract}

\vspace{1cm}

\begin{center}
    \textbf{Hints}
\end{center}

\textbf{First part}: Instead of using the equation of a line as $y-y_0=m(x-x_0)$ use $z-y_0=m(t-x_0)$. 

\vspace{1cm}

\textbf{Second part}:  Find the length of $PQ$.

\vspace{1cm}

\textbf{Third part}: What must be true in order for $(0,2)$ to be the minimum point of the curve $y=\mathrm{f}(x)$? Can you then use this to eliminate one of the possible values of $\mathrm{f}'(x)$?



\newpage

\begin{center}
    \textbf{Solution}
\end{center}

\vspace{0.5cm}

Differentiating the curve $y=\mathrm{f} (x)$,
\[
\frac{\mathrm{d}y}{\mathrm{d}x}= \mathrm{f}'(x)
\]
gives us the expression for the gradient of the normal at $P:(x,\mathrm{f}(x))$ as $m_p=-1 / \mathrm{f}'(x)$. Using the equation of a straight line,
\[
\text{Normal at $P$ }: z-\mathrm{f}(x) = -\frac{1}{\mathrm{f}'(x)} (t-x)
\]
where instead of the equation of the line being $y-y_0=m(x-x_0)$, we have used $z-y_0=m(t-x_0)$ as we have already used the variable $x$ in our point $P$. This leaves us with,
\[
\text{Normal at $P$ }: z = -\frac{1}{\mathrm{f}'(x)}t + \frac{x}{\mathrm{f}'(x)}+\mathrm{f}(x)
\]
giving 
\[
Q:\left(0 ,  \mathrm{f}(x)+\frac{x}{\mathrm{f}'(x)}\right).
\]

\vspace{0.5cm}

Using pythagoras' theorem to find the length $PQ$,
\[
PQ^2 = (x-0)^2 + \left(\frac{x}{\mathrm{f}'(x)}+\mathrm{f}(x) - \mathrm{f}(x)\right)^2 = x^2 + \left(\frac{x}{\mathrm{f}'(x)}\right)^2
\]
and equating it with the given,
\[
x^2 + \left(\frac{x}{\mathrm{f}'(x)}\right)^2 = x^2 + e^{x^2} \Rightarrow (\mathrm{f}'(x))^2 =  x^2e^{-x^2}.
\]
We can solve for $\mathrm{f}'(x)$,
\[
\mathrm{f}'(x)=  \pm xe^{-x^2/2}
\]
and find $\mathrm{f}''(x)$,
\[
\mathrm{f}''(x)=  \pm e^{-x^2/2} \mp x^2e^{-x^2/2}
\]
which evaluated at $(0,2)$,
\[
\mathrm{f}''(0)=  \pm 1
\]
which has to be greater than zero as we are given that $(0,2)$ is a minimum.\par 
\quad Hence,
\[
\mathrm{f}'(x)=   xe^{-x^2/2}
\]
and integrating,
\[
\mathrm{f}(x) = C-e^{-x^2/2}
\]
where $C$ is an arbitrary constant. Using the point $(0,2)$ to find $C$ leaves us with 
\[
\mathrm{f}(x) = -1-e^{-x^2/2}.
\]
\newpage

\begin{center}
    \textbf{Discussion}
\end{center}

\vspace{0.5cm}

As far as STEP questions go, this one is quite straight forward. Arguably the hardest part is in the initial equation of the normal through $P$ and finding out which $\mathrm{f}'(x)$ to use. The phrasing of the question, writing $P:(x,\mathrm{f}(x))$, provides discomfort when attempting to use the traditional equation of the line $y-y_0=m(x-x_0)$ as the variable $x$ is already in use, yet a simple change of notation solves this problem. When attempting to form the differential equation, we arrive at 
\[
\left(\frac{\mathrm{df}}{\mathrm{d}x}\right)^2 = x^2e^{-x^2}
\]
and we have to deal with which root to pick, the positive or the negative? Using the given minimum leads to the use of 
\[
\mathrm{f}'(x) = xe^{-x^2/2}.
\]
Overall, this question is quite straightforward and should be accessible to all those who don't mind some initial discomfort. 
\end{document}
