\documentclass{article}
\usepackage{graphicx}
\usepackage{amsmath}   
\usepackage{amsthm}
\usepackage{amsfonts} 
\usepackage{tikz}
\usepackage[hidelinks]{hyperref}
\usepackage{pgfplots}
\usepackage{parskip}
\pgfplotsset{compat=1.18}
\setlength{\parindent}{0pt}
\renewcommand{\baselinestretch}{1.3}

% Macros 
\newcommand\bfrac[2]{\frac{\displaystyle #1}{\displaystyle #2}}

\title{96-S1-Q7 \\ Solution + Discussion}
\author{David Puerta}
\date{}

\definecolor{BLUE}{RGB}{101,149,179}

\newcommand{\der}[2]{\frac{\mathrm{d}#1}{\mathrm{d}#2}}

\begin{document}

\maketitle

\begin{abstract}
    \noindent In this document we will go through the solution to the 96-S1-Q7 question and provide a discussion of the question at the end. There are also hints on the first page to aid you in finding a solution. There is no single method that results in an answer to a STEP question, there are a multitude of different paths that end up at the same solution. However, some methods are more straight forward and you are encouraged to take the path of least resistance.  
\end{abstract}

\vspace{1cm}

\begin{center}
    \textbf{Hints}
\end{center}

\textbf{(i)}: Start from $\der{y}{t} \propto - y$. You want to write $e^{-kt}=b^t$, can you use any exponent laws? Don't forget to show that $b<1$.

\vspace{1cm}

\textbf{(ii)}: Add a $+a$ into your differential equation from part (i).



\newpage

\begin{center}
    \textbf{Solution}
\end{center}

\vspace{0.5cm}

\textbf{(i)} Since the water flows \textit{out} of the tank at a rate proportional to the amount of water in the tank, we have 
\[
\der{y}{t} \propto - y
\]
where the negative sign comes from the water flowing \textit{out} of the tank, if it was the water flowing \textit{in} it would be positive. Hence, we have 
\[
\der{y}{t}=-ky
\]
for some $k>0$. Solving the differential equation, we arrive at 
\[
y(t)=Ae^{-kt}
\]
for some constant $A$. We can then use the initial condition $y(0)=1$ to get 
\[
y(0)=A=1 \Rightarrow y(t)=e^{-kt}
\]
To get it in the form $y=b^t$, observe that 
\[
e^{-kt} = (e^{-k})^t 
\]
and letting $b=e^{-k}$ we get 
\[
y(t)=b^t
\]
Finally, since $k>0$, we have that $e^k>1$ and thus $b = \frac{1}{e^k}<1$.

\textbf{(ii)} Now we are adding water at a constant rate $a$, which gives us the new differential equation 
\[
\der{y}{t} = \underbrace{-ky}_{\text{out}}+\underbrace{a}_{\text{in}}
\]
for $k,a>0$. Since $b=e^{-k}$ we have that $-k = \ln(b)$ and so,
\[
\der{y}{t} -y\ln(b)=a
\]
We can now find the integrating factor 
\[
I = \exp\left(\int -\ln(b)\,\mathrm{d}t\right) = b^{-t}
\]
and multiplying our differential equation by it,
\[
b^{-t}\bigg(\der{y}{t} -y\ln(b)=a\bigg) 
\]
we see that this can be written as 
\[
\der{}{t}(yb^{-t}) = ab^{-t} \Rightarrow yb^{-t}=\int ab^{-t}\,\mathrm{d}t
\]
Observe that,
\[
\der{}{t}(b^{-t}) = \der{}{t}\bigg(\left(b^{t}\right)^{-1}\bigg) = -(b^{t})^{-2}\cdot b^t\ln(b) = -\ln(b)b^{-t}
\]
which we can use to find that 
\[
\int ab^{-t} \,\mathrm{d}t = -\frac{ab^{-t}}{\ln(b)}+c
\]
As a result,
\[
yb^{-t} = -\frac{ab^{-t}}{\ln(b)}+c
\]
using the initial condition $y(0)=1$,
\[
c = 1 + \frac{a}{\ln(b)}
\]
which gives 
\[
y(t) = \left(1+\frac{a}{\ln(b)}\right)b^t - \frac{a}{\ln(b)}
\]
\newpage

\begin{center}
    \textbf{Discussion}
\end{center}

\vspace{0.5cm}

A surprisingly straight forward STEP question, if you are comfortable with solving separable and integrating factor differential equations you will find this question quite easy. 

In part (i), the hardest part is probably forming the differential equation 
\[
\der{y}{t} = -ky
\]
and writing $e^{-kt}$ as $b^{t}$. When forming a differential equation, remember that any loss to the system shows up as a negative and any gain shows up as a positive; these two facts give rise to the inequality $k>0$ which is needed to show that $b<1$. 

For part (ii), the hardest part would be forming the differential equation yet using the tip above leads to a very quick answer. After that, make sure you remember how to use the integrating factor method. 

The only thing you may not have seen before in this question is computing an integral in the form of 
\[
\int b^{nt} \, \mathrm{d}t
\]
where the best way to approach this is to compute the derivative 
\[
\der{}{t}\bigg(b^{nt}\bigg)= nb^{nt}\ln(b)
\]
using the method shown in the solution, and then rearranging for a total derivative 
\[
\der{}{t}\bigg(\frac{b^{nt}}{n\ln(b)}\bigg)= b^{nt}
\]
giving 
\[
\int b^{nt} \, \mathrm{d}t = \frac{b^{nt}}{n\ln(b)} + c
\]
If you are already confident with forming and solving differential equations, this question will feel a bit too easy for a STEP question.
 \end{document}
