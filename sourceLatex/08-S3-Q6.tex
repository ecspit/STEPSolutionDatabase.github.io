\documentclass{article}
\usepackage{graphicx}
\usepackage{amsmath}   
\usepackage{amsthm}
\setlength{\parindent}{0pt}

\title{08-S3-Q6 \\ Solution + Discussion}
\author{David Puerta}
\date{}

\begin{document}

\maketitle

\begin{abstract}
    \noindent In this document we will go through the solution to the 08-S3-Q6 question and provide a discussion of the question at the end. There are also hints on the first page to aid you in finding a solution. There is no single method that results in an answer to a STEP question, there are a multitude of different paths that end up at the same solution. However, some methods are more straight forward and you are encouraged to take the path of least resistance.  
\end{abstract}

\vspace{1cm}

\begin{center}
    \textbf{Hints}
\end{center}

\textbf{Part (i)}: Differentiate $p$ implicitly just as you would differentiate $y$.

\vspace{1cm}

\textbf{Part (ii)}:  Differentiate with respect to $x$.


\newpage

\begin{center}
    \textbf{Solution}
\end{center}

\vspace{0.5cm}

\textbf{(i)} Differentiating with respect to $x$,
\[
\frac{\mathrm{d}}{\mathrm{d}x} \left[y = p^2 + 2xp \right] 
\]
which leaves us with 
\[
\frac{\mathrm{d}y}{\mathrm{d}x} = \frac{\mathrm{d}}{\mathrm{d}x} (p^2 + 2xp).
\]
As $p$ is a function of $x$ we can implicitly differentiate the right hand side of the above equation
\[
\frac{\mathrm{d}y}{\mathrm{d}x} = 2p  \frac{\mathrm{d}p}{\mathrm{d}x} + 2x \frac{\mathrm{d}p}{\mathrm{d}x} + 2p 
\]
and noting that $p = \frac{\mathrm{d}y}{\mathrm{d}x}$ we can rearrange to arrive at 
\[
\frac{\mathrm{d}x}{\mathrm{d}p} = -2 - \frac{2x}{p}.
\]
Notice that we can rearrange the equation to 
\[
\frac{\mathrm{d}x}{\mathrm{d}p} + \frac{2x}{p} = -2
\]
which is a first order differential equation which can be solved with the integrating factor
\begin{equation*}
I = \exp \left( \int \frac{2}{p} \,\mathrm{d}p \right) = \exp (2 \ln (p)) = p^2.
\end{equation*}
Multiplying the differential equation by $I$,
\[
\frac{\mathrm{d}}{\mathrm{d}p}[xp^2] = -2p^2
\]
and after integrating both sides we arrive at our desired
\[
x = -\frac{2}{3}p+Ap^{-2}
\]
where $A$ is an arbitrary constant.\par

\quad Using our initial conditions ($p=-3$ when $x=2$) we find that
\[
2 = -\frac{2}{3}(-3)+\frac{A}{9} \Rightarrow A = 0
\]
and thus
\[
x = -\frac{2}{3}p.
\]
Replacing $p$ with $\frac{\mathrm{d}y}{\mathrm{d}x}$,
\begin{equation*}
x = -\frac{2}{3}\frac{\mathrm{d}y}{\mathrm{d}x}
\end{equation*}
which can be solved to get
\begin{equation*}
y(x) = -\frac{3}{4}x^2 + C
\end{equation*}
where $C$ is an arbitrary constant.\par
\quad Using the initial conditions in the original equation $y=p^2+2xp$ we get that $y=-3$ when $x=2$. Hence we have
\[
-3 = -\frac{3}{4}(4) + C \Rightarrow C = 0
\]
and thus
\[
y(x) = -\frac{3}{4}x^2.
\]

\vspace{0.5cm}

\textbf{(ii)} Using the same idea as part (i), differentiating with respect to $x$,
\begin{align*}
\frac{\mathrm{d}y}{\mathrm{d}x} = \frac{\mathrm{d}}{\mathrm{d}x}(2xp+p\ln(p)),
\end{align*}
using the chain and product rule,
\[
\frac{\mathrm{d}y}{\mathrm{d}x} = 2p + 2x\frac{\mathrm{d}p}{\mathrm{d}x} + \frac{\mathrm{d}p}{\mathrm{d}x}\ln(p) + p \frac{1}{p}\frac{\mathrm{d}p}{\mathrm{d}x}
\]
and noting that $p = \frac{\mathrm{d}y}{\mathrm{d}x}$ we can rearrange the equation to 
\[
\frac{\mathrm{d}x}{\mathrm{d}p} = -\frac{1}{p} -\frac{2x}{p} - \frac{\ln(p)}{p}
\]
which can be written as
\[
\frac{\mathrm{d}x}{\mathrm{d}p} + \frac{2x}{p} = -\frac{1}{p} - \frac{\ln(p)}{p}.
\]
This is again a first order differential equation for $x$ in terms of $p$ that can be solved with the same integrating factor ($I = p^2$) as before. Multiplying by $I$,
\[
\frac{\mathrm{d}}{\mathrm{d}p}[xp^2] = -p-p\ln(p).
\]
Integrating both sides,
\begin{align*}
& xp^2 = \int -p-p\ln(p) \,\mathrm{d}p = - \int p \,\mathrm{d}p - \int p\ln(p) \,\mathrm{d}p \\
& \Rightarrow xp^2 = -\frac{p^2}{2} - \left[\frac{p^2}{2}\ln(p) - \int \frac{1}{p} \cdot \frac{p^2}{2} \,\mathrm{d}p\right] = -\frac{p^2}{2}\left(\frac{1}{2}+\ln(p)\right) + B
\end{align*}
where $B$ is an arbitrary constant. Hence we arrive at
\[
x = -\frac{1}{2}\left[\frac{1}{2} + \ln(p) \right] + Bp^{-2}.
\]
Using our initial conditions ($p=1$ when $x= -\frac{1}{4}$) we see that
\[
-\frac{1}{4} = -\frac{1}{4} + B \Rightarrow B=0
\]
thus
\[
x = -\frac{1}{2}\ln(p) - \frac{1}{4}.
\]

Rearranging for $p$ in terms of $x$ and using $p = \frac{\mathrm{d}y}{\mathrm{d}x}$,
\[
p = \frac{\mathrm{d}y}{\mathrm{d}x} = e^{-2x-\frac{1}{2}}
\]
and integrating both sides,
\[
y = -\frac{1}{2}e^{-2x-\frac{1}{2}} + C.
\]
From $y=2xp+p\ln(p)$ and our initial conditions ($p=1$ when $x=-\frac{1}{4}$) we have that $y=-\frac{1}{2}$ when $x=-\frac{1}{4}$. Solving for $C$,
\[
-\frac{1}{2} = -\frac{1}{2}e^{0} + C \Rightarrow C=0
\]
and hence we have
\[
y = -\frac{1}{2}e^{-2x-\frac{1}{2}}.
\]

\newpage

\begin{center}
    \textbf{Discussion}
\end{center}

\vspace{0.5cm}

As far as STEP question goes this one is pretty straight forward. If you ignore the initial discomfort of having $p=\mathrm{d}y / \mathrm{d}x$ and the having $\mathrm{d}x / \mathrm{d}p$ (differentiating $x$ as a function of $\mathrm{d}y / \mathrm{d}x$!) then the actual computation is clear-cut. To understand why viewing the differential equations
\begin{equation*}
y = \left(\frac{\mathrm{d}y}{\mathrm{d}x} \right)^2 + 2x \frac{\mathrm{d}y}{\mathrm{d}x} \quad \text{and} \quad  y=2x \frac{\mathrm{d}y}{\mathrm{d}x} + \frac{\mathrm{d}y}{\mathrm{d}x} \ln \left(\frac{\mathrm{d}y}{\mathrm{d}x}\right)
\end{equation*}
using the substitution $p=\frac{\mathrm{d}y}{\mathrm{d}x}$ works in the first place is to see when it made the question easier.\par

\quad Starting with
\begin{equation*}
y = \left(\frac{\mathrm{d}y}{\mathrm{d}x} \right)^2 + 2x \frac{\mathrm{d}y}{\mathrm{d}x}
\end{equation*}
we see that in our solution we arrive at
\begin{equation*}
x = -\frac{2}{3}p+Ap^{-2}
\end{equation*}
with $A$ an arbitrary constant. If we remember what we are after ($y$ as a function of $x$), we see that we would like to find a solution for $p$ in terms of $x$ and integrate the result to find $y$. Yet this is only (reasonably) possible if we allow for initial conditions such that $A=0$ - which is what we had in the question. If instead we require that $A \neq 0$ we arrive at a standstill as there is no reasonable way to solve for $p$.\footnote{It should be noted that the equation can be written as a cubic in $p$ yet this leads to some ugly integration, try it yourself!} 
Similarly with
\begin{equation*}
y=2x \frac{\mathrm{d}y}{\mathrm{d}x} + \frac{\mathrm{d}y}{\mathrm{d}x} \ln \left(\frac{\mathrm{d}y}{\mathrm{d}x}\right)
\end{equation*}
we see that in our solution we arrive at
\begin{equation*}
x = -\frac{1}{4} - \frac{1}{2}\ln(p) + Bp^{-2}
\end{equation*}
with $B$ an arbitrary constant. We see that if $B\neq 0$ we are in the same situation as part (i) - arguably worse as we have $\ln(p)$.\par

\quad Overall, this question is accessible to all those who can follow instructions, comfortable with integrating factors and smile as everything falls into place.


\end{document}
