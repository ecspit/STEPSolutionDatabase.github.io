\documentclass{article}
\usepackage{graphicx}
\usepackage{amsmath}   
\usepackage{amsthm}
\usepackage{tikz}
\usepackage{pgfplots}
\pgfplotsset{compat=1.18}
\setlength{\parindent}{0pt}

\title{93-S3-Q12 \\ Solution + Discussion}
\author{David Puerta}
\date{}

\definecolor{BLUE}{RGB}{101,149,179}

\begin{document}

\maketitle

\begin{abstract}
    \noindent In this document we will go through the solution to the 93-S3-Q12 question and provide a discussion of the question at the end. There are also hints on the first page to aid you in finding a solution. There is no single method that results in an answer to a STEP question, there are a multitude of different paths that end up at the same solution. However, some methods are more straight forward and you are encouraged to take the path of least resistance.  
\end{abstract}

\vspace{1cm}

\begin{center}
    \textbf{Hints}
\end{center}

\textbf{First part}: Draw a diagram and use the fact that the system is in equilibrium.

\vspace{1cm}

\textbf{Second part}: Draw a diagram and use conservation of energy.

\vspace{1cm}

\textbf{Third part}: Show that $T_Q - T_R >0$.



\newpage

\begin{center}
    \textbf{Solution}
\end{center}

\vspace{0.5cm}

Drawing a diagram,

\begin{figure}[h!]
\centering
\begin{tikzpicture}[scale=1]
    \node[above] at (-4,0) {$A$};
    \fill[black] (-4,0) circle (1pt);
    
    \node[above] at (-2,0) {$B$};
    \fill[black] (-2,0) circle (1pt);
    
    \node[above] at (2,0) {$C$};
    \fill[black] (2,0) circle (1pt);

    \draw[<->] (2,-0.1) -- (2,-2);
    \node[right] at (2,-1) {$\frac{a}{4}$};
    
    \node[above] at (4,0) {$D$};
    \fill[black] (4,0) circle (1pt);

    \node[above] at (-3,0.2) {$a$};
    \draw[<->] (-3.8,0.2) -- (-2.2,0.2);
    
    \node[above] at (3,0.2) {$a$};
    \draw[<->] (3.8,0.2) -- (2.2,0.2);

    \node[above] at (-1,0.2) {$3a$};
    \draw[<->] (-1.8,0.2) -- (-0.05,0.2);

    \node[above] at (1,0.2) {$3a$};
    \draw[<->] (1.8,0.2) -- (0.05,0.2);

    \draw[] (-4,0) -- (-2,0);
    \draw[] (2,0) -- (4,0);
    \draw[] (-2,0) -- (0,-2);
    \draw[->] (0,-2) -- (-1,-1);
    \draw[] (0,-2) -- (2,0);
    \draw[->] (0,-2) -- (1,-1);
    \draw[dotted] (0,-2) -- (0,0);

    \node[below] at (-0.2,-2.1) {$P$};
    \node[below] at (0,-3) {$mg$};
    \draw[->] (0,-2) -- (0,-3);
    \fill[BLUE] (0,-2) circle (3pt);

    \node[right,below] at (1.2,-1) {$T$};
    \node[left,below] at (-1.2,-1) {$T$};

    \draw[-] (0,-1.5) arc[start angle=90, end angle=119.87, radius=0.8];
    \node at (-0.3,-1.3) {$\theta$};
\end{tikzpicture}
\end{figure}

we can calculate the tension in each string as 
\[
T = \frac{\lambda x}{l} = \frac{kmg}{2a} \left(\frac{a\sqrt{145}}{2}\right) = \frac{1}{4}kmg\sqrt{145}
\]
and as the system is in equilibrium,
\[
2T\cos \theta  = mg
\]
where $\theta$ is the angle of the tension of the string with the vertical.\par 
\quad Hence,
\[
2 \times \frac{1}{4}kmg\sqrt{145} \times \frac{1}{\sqrt{145}} = mg \Rightarrow k=2
\]
using some trigonometry.\par

\quad Beginning with a diagram,

\begin{figure}[h!]
\centering
\begin{tikzpicture}[scale=0.8]
    \node[above] at (-4,0) {$A$};
    \fill[black] (-4,0) circle (1pt);
    
    \node[above] at (-2,0) {$B$};
    \fill[black] (-2,0) circle (1pt);
    
    \node[above] at (2,0) {$C$};
    \fill[black] (2,0) circle (1pt);

    \draw[<->] (0,-0.05) -- (0,-1.9);
    \node[right] at (0,-1) {$pa$};

    \draw[<->] (0,0.05) -- (0,3.9);
    \node[right] at (0,2) {$3a$};
    
    \node[above] at (4,0) {$D$};
    \fill[black] (4,0) circle (1pt);

    \draw[] (-4,0) -- (-2,0);
    \draw[] (2,0) -- (4,0);
    \draw[] (-2,0) -- (0,-2);
    \draw[] (0,-2) -- (2,0);

    \draw[dotted] (-4,0) -- (0,4);
    \draw[dotted] (0,4) -- (4,0);

    \node[below] at (-0.2,-2.1) {$Q$};
    \fill[BLUE] (0,-2) circle (3pt);

    \node[above,left] at (-0.2,4.1) {$R$};
    \fill[BLUE] (0,4) circle (3pt);    
\end{tikzpicture}
\end{figure}

we will consider the energies at point $Q$,
\begin{align*}
& \text{At $Q$ : }\quad  x_{\mathrm{total}} = 2a\sqrt{9+p^2},\quad  E_e = \frac{\lambda x_{\mathrm{total}^2}}{2l} = 2mga(9+p^2) \ \mathrm{J}\\ 
& \Rightarrow E_{\mathrm{total}} = 2mga(9+p^2) \ \mathrm{J}
\end{align*}
and at point $R$,
\begin{align*}
& \text{At $R$ : }\quad  x_{\mathrm{total}} = 8a, \quad  E_e = \frac{\lambda x_{\mathrm{total}^2}}{2l} = 32mga \ \mathrm{J}, \quad E_p = mga(p+3) \ \mathrm{J} \\ 
& \Rightarrow E_{\mathrm{total}} = mga(35+p) \ \mathrm{J}.
\end{align*}
Using conservation of energy,
\[
2mga(9+p^2)= mga(35+p)
\]
which gives 
\[
2p^2-p-17=0.
\]
Solving the quadratic gives $p=(1 \pm \sqrt{137})/4$ and as $p$ is positive we have $p=(1 +\sqrt{137})/4$.To show the tension at $Q$ is greater than the tension at $R$ we will show that the extension at $Q$ is greater than the extension at $R$.\par 

\quad Looking at the difference between the extensions,
\[
x_Q-x_R = 2a\sqrt{p^2+9} - 8a = 2a\{\sqrt{p^2+9}-4\}
\]
and we see that 
\[
p^2 = \frac{69}{8} + \frac{\sqrt{137}}{8} > \frac{64}{8} = 8 >7 \Rightarrow p^2+9>16.
\]
Thus,
\[
x_Q-x_R  = 2a\{\sqrt{p^2+9}-4\} > 2a\{\sqrt{16}-4\} = 0 \Rightarrow x_Q > x_R.
\]


\newpage

\begin{center}
    \textbf{Discussion}
\end{center}

\vspace{0.5cm}

If you are attempting a mechanics question without a diagram of the system then you are tying the carriage in front of the horses. First and foremost, draw a diagram. Once you have a diagram, the first part is straight forward once you resolve the tension in each string with the weight.\par
\quad The second part requires a new diagram showing the middle of the string now at point $R$. Calculating the total extension at $Q$ and $R$ then using conservation of energy yields the required quadratic.\par
\quad Finally, the third part requires showing that the tension at $Q$ is greater than the tension at $R$. If our goal is to show that 
\[
T_Q>T_R
\]
it is often easier to equivalently show that
\[
T_Q-T_R>0
\]
and this subtlety often leads to a more simple show that answer. Once set up, all that is left is some trivial inequality work to show that $T_Q-T_R>0$.\par
\quad Overall, this question relies on a clear, accurate diagram with well laid out workings.

\end{document}
