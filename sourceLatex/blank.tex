\documentclass{article}
\usepackage{graphicx}
\usepackage{amsmath}   
\usepackage{amsthm}
\usepackage{tikz}
\usepackage{pgfplots}
\pgfplotsset{compat=1.18}
\setlength{\parindent}{0pt}

\title{XX-SX-QX \\ Solution + Discussion}
\author{David Puerta}
\date{}

\definecolor{BLUE}{RGB}{101,149,179}

\begin{document}

\maketitle

\begin{abstract}
    \noindent In this document we will go through the solution to the XX-SX-QX question and provide a discussion of the question at the end. There are also hints on the first page to aid you in finding a solution. There is no single method that results in an answer to a STEP question, there are a multitude of different paths that end up at the same solution. However, some methods are more straight forward and you are encouraged to take the path of least resistance.  
\end{abstract}

\vspace{1cm}

\begin{center}
    \textbf{Hints}
\end{center}

\textbf{First part}:

\vspace{1cm}

\textbf{Second part}: 



\newpage

\begin{center}
    \textbf{Solution}
\end{center}

\vspace{0.5cm}


\newpage

\begin{center}
    \textbf{Discussion}
\end{center}

\vspace{0.5cm}

\end{document}
